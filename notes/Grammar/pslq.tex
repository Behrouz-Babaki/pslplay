\documentclass[12pt]{article} 

\usepackage{syntax}

\begin{document}
\title{$\mathit{PSL}^{Q}$ Grammar}
\date{}
\maketitle

\paragraph{Syntax Notation}
\begin{itemize}
\item Non-terminals are written between angle brackets,
  e.g. $\langle\mathit{formula}\rangle$.
\item Terminals are written in fixed-width font enclosed in quotation
  signs, e.g. `\verb+Q+'.
\item Regular expressions are written in fixed-width font, e.g. \verb#[0-9]+#.
\item Optional items are written in square brackets, e.g. $[$
  \verb#0+# $]$.
\item Non-empty lists are written with an item, followed by a
  separator and ``\ldots'' , e.g. $\langle \mathit{rule} \rangle$ `\verb#;#' \ldots 

\end{itemize}

\paragraph{Grammar Definition}

\begin{grammar}

  <program> ::= <rule> `;' \ldots
  
  % <rules> ::= <rules> <rule>
  % \alt <rule>

  <rule> ::= <weight> `:' <conj_formula> `=>' <disj_formula>

  <weight> ::= \verb#[0-9]+# $[$ `.' \verb#[0-9]+# $]$
  
  <conj_formula> ::= <formula>
  \alt <conj_formula> `\&' <formula>

  <disj_formula> ::= <formula>
  \alt <disj_formula> `|' <formula>

  <formula> ::= <predicate>
  \alt <quantifier_expr>
  \alt <constant_value>
  \alt `(' <formula> `)'
  \alt `~' <formula>

  <predicate> ::= <predicate_symbol> `(' <term> `,' \ldots `)'

  <predicate_symbol> ::= \verb#[a-zA-Z][a-zA-Z0-9_]*#

  <term> ::= <variable_name> 
  \alt <constant_term>

  <variable_name> ::= \verb#[A-Z][a-zA-Z0-9_]*#
  
  <constant_term> ::= \verb#[a-z][a-zA-Z0-9_]*#

  <quantifier_expr> ::= `Q' `(' <variable_name> `,' <formula> `,' <formula> `)'

  <constant_value> ::= \verb#0+# $[$ `.' \verb#[0-9]+# $]$ 
  \alt $[$ \verb#0+# $]$ `1' $[$ `.' \verb#0+# $]$
  
\end{grammar}

Since the symbol `\verb#Q#' is reserved for quantifier expressions, a predicate
symbol can not be the symbol `\verb+Q+'.




\end{document}